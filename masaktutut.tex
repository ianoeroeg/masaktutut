\documentclass[a4paper]{article}
\usepackage{multicol,lipsum}
\usepackage[a4paper,includeheadfoot,margin=1.5cm]{geometry}
\begin{document}
\title{Tutut Brewing Process}\author{Nusirwan Hakim}
\maketitle
\begin{center}
%\tableofcontents
\textbf{Brewing Tutut..}

\emph{Brewing coffee mah biasa, tapi kalo brewing tutut itu terbilang istimewa.
Kenapa istimewa, karena jika memprosesnya salah, bukannya kenikmatan yang diterima melainkan keracunan makanan. Istimewanya juga adalah, ketika kita sudah selesai dan mampu brewing tutut, tidak ada yang bisa dibanggakan, apalagi nyinyirin mereka yang makan tutut yang digunting seperti para neo coffee enthusiast, karena tutut tidak usah digiling apalagi digunting. Karena makan tutut itu buat yang tidak mampu beli daging\footnote{https://www.tribunnews.com/nasional/2017/12/06/daging-sapi-mahal-menteri-pertanian-tawarkan-keong-sawah-sebagai-alternatif-netizen-heboh?page=all}, katanya, jadi apa yang dibanggakan dengan masak tutut coba? Kere!!!}\end{center}
\begin{multicols}{2}
\section{Pendahuluan}
Brewing tutut bisa dibilang gampang gampang susah, karena dimulai dari tahap ngala tutut, dilanjutkan dengan proses pembersihan tutut dari kotorannya baik kotoran 'e'e si tutut itu sendiri atau pun yang menempel di badan tutut. Setelah bersih, tahap berikutnya adalah pemotongan moncong dari tutut, ini sangat leukleuk karena harus dipotong satu per satu. Berikutnya adalah proses pembersihan kembali dengan cara direbus. Dilanjutkan dengan persiapan bumbu tutut. dan terakhir proses memasaknya sampai selesai.

Sebelum melakukan proses konsumsi tutut, berdo'a terlebih dahulu kepada Allah agar diberikan kebaikan dari tutut ini dan dijauhkan dari keburukan yang bisa diakibatkan dari tutut ini. Jangan lupa adab-adab berdo'anya juga.

\section{Ngala tutut}
Untuk memperoleh tutut anda dapat mencari sendiri di kolam ikan yang anda punya sendiri, atau kalau anda punya sawah sendiri, atau juga kalau anda dekat dengan area rawa atau bendungan seperti saguling. Jadi untuk rekan-rekan yang rumahnya di kotabaru parahyangan, meskipun anda adalah orang kaya tapi tetep pengen merasakan sensasi ngala tutut, pergi saja ke pinggiran2 saguling yang banyak eceng gondoknya, biasanya suka ketemu banyak tutut, tapi jangan serakah, kasihan para peternak atau nelayan yang salah satu pencahariannya diperoleh dari tutut. Untuk ngala tutut, anda cukup punya jaring atau ayakan, cari area yang kira-kira tumbuh tutut yang banyak, gunakan ayakan/jaring untuk mengeruk bagian lumpurnya, kemudian pisahkan tutut dari lumpur. Simpan di wadah yang sudah anda siapkan.

\section{Pembersihan tahap I (kotoran internal)}
Tutut yang sudah kita tangkap, biasanya masih memiliki kotoran dari proses pencernaan yang mereka lakukan (meureun). Untuk membuang kotoran ini, siapkan air bersih setidaknya 3x volume tutut yang ditangkap, lebih banyak lebih baik. Masukkan air bersama tutut yang sudah ditangkap tadi, dan diamkan sekurang-kurangnya selama 24 jam. Kalau perlu, ember atau bejana yang digunakan untuk menyimpan tutut ini kita tutup dengan saringan atau ayakan juga boleh, untuk menghindari kaburnya tutut dari tempat pembersihan.

\section{Pembersihan tahap II (kotoran eksternal)}
Setelah tutut didiamkan dalam air bersih selama 1 atau dua hari, tiriskan tutut dan pisahkan dari air. Berikutnya isi kembali tutut dengan air dengan tujuan untuk membersihkan tutut dari kotoran yang menempel di cangkang tututnya seperti lumut atau lainnya. Lakukan pembersihan dengan cara seperti ngisikan beas, dengan melakukan putaran sampai air yang digunakan menjadi warna kehijauan atau lebih gelap lagi kemudian buang airnya. ulangi langkah ini hingga air yang dihasilkan tidak kotor.

\section{Pemotongan Moncong Tutut}
Entahlah namanya apa, yang pasti moncong yang dimaksud ini adalah bagian ujung, atau bagian apex dari cangkang tutut. Untuk melakukan pemotongan ini, anda bisa menggunakan golok, kapak, atau pisau, tapi penggunaan pisau tidak disarankan. Lakukan pemotongan bagian apex ini maksimal sampai lingkar atau junction spiral  antar jalinan cangkang tutut yang kedua. Proses pemotongan ini terbilang penting karena akan menentukan mudah atau tidaknya anda nanti ngecrokan tututna ketika sudah siap dikonsumsi.

\section{Pembersihan tahap III}
Pembersihan tahap III ini sebenarnya adalah proses perebusan awal tutut. Pada proses perebusan awal ini, seringkali air rebusan menjadi hijau atau muncul seperti buih. Jika diperoleh kondisi ini, buang air dan buih yang dihasilkan kemudian ganti dengan air yang baru. Ulangi langkah ini jika masih dihasilkan kondisi yang sama.

\section{Persiapan bumbu tutut}
Bumbu serta bahan yang saya gunakan untuk tutut brewing ini adalah :
\begin{itemize}\item Tutut : 248 ekor tutut (itung aja pas motongin moncongnya, sugan gelo) Jumlah tutut sesuai kebutuhan, dua rawuan tangan aja lah.
\item Kunyit : 6 cm kubik
\item Jahe : 4 cm kubik
\item Bawang Putih : 3 sampai 4 siung
\item Bawang Merah : 3 sampai 4 siung
\item Daun Salam : 3 sampai 4 lembar
\item Sereh : satu saja, memarkan bagian bawahnya.
\item Cengek Domba : 4 buah atau sesuai selera, pakai jolokia bhut juga lumayan enak.
\item Garam : sesuai selera
\item Gula : sesuai selera tinggal dikombinasikan dengan garam
\end{itemize}
Optional : kalau anda doyang mecin, bisa digunakan sebagai pengganti gula, atau kalau anda doyan mecin tapi tidak doyan disebut generasi mecin, pake aja penyedap masakan seperti maskoko, royo atau sesamanya.

Gunakan coet kemudian satukan semua bahan dengan cara direndos pake mutu, sesuaikan dengan selera, grind kasar juga sebenarnya juga sudah cukup, tapi kalau anda lebih suka fine grind atau medium grind, itu adalah pilihan anda. Saya sarankan anda gunakan coet atau mutu dari batu, jangan gunakan blender, karena ditakutkan anda tidak memperoleh sensasi rasa batu yang tergerus dengan bumbu.
Jika sudah selesai digerus, terdapat dua pilihan sebelum dimasukkan ke dalam rebusan tutut. Bumbu bisa anda tumis dulu sebentar sampai aroma masakannya tercium atau bisa anda langsung campurkan nanti dengan rebusan tutut.

\section{Final Brew}
Rebus tutut dengan air, dengan perbandingan air dan tutut sebesar 2:1, di mana airnya adalah 2x banyaknya tutut. Api yang digunakan adalah api sedang, atau jika anda sanggup, gunakan bara  api untuk hasil yang lebih nikmat (meureun eta oge, ngan ari percobaan saya mah kitu).
Jika air sudah mendidih, masukkan bumbu yang sudah disiapkan sebelumnya (yang ditumis atau langsung) kemudian kocek agar bercampur dengan airnya. Masak dalam air mendidih hingga sekitar 5-10 menit, kemudian turunkan api menjadi api kecil atau cukup dengan mode bara api dan biarkan selama kurang lebih setengah jam, kemudian matikan api atau angkat bejana dari tungku.
Sesaat setelah tutut diangkat dari atas api, jangan langsung konsumsi tutut, karena berbahaya, \textbf{PANAS KENEH!} Tunggu hingga suhu air atau tututnya mencapai suhu yang wajar untuk dikonsumsi, atau sanggup anda konsumsi.
\end{multicols}
\begin{center}\emph{\textbf{Tutut rebus a la om ian siap disajikan.}}\end{center}
\section{Disclaimer}
\textbf{penulis tidak bersedia dimintai pertanggung jawaban di depan hukum terkait hasil yang diperoleh jika anda mengikuti catatan ini, baik ternyata hasilnya membuat anda nikmat atau pun tidak}
\end{document}
